\documentclass[12pt,a4paper,sans]{moderncv} 
\moderncvstyle{fancy}
\moderncvcolor{green}
\nopagenumbers{}
\usepackage[utf8]{inputenc} 
\usepackage{CJK}
\recomputelengths 
\setmainfont{FandolHei}
\setsansfont{FandolHei}
\CJKtilde

\usepackage[scale=0.98]{geometry}
\name{李}{杨}
\title{Android系统软件开发}
\phone[mobile]{\color{black}{\textbf{18513853385  18410197201}}}
\email{liyangzmx@163.com}
\photo[64pt][0.4pt]{picture.png}

\definecolor{lgreen}{HTML}{59b24c}

\makeatletter\renewcommand*{\bibliographyitemlabel}{\@biblabel{\arabic{enumiv}}}\makeatother
\begin{document}

\makecvtitle

\section{个人信息}
\cvitem{职位}{Android驱动工程师}
\cvitem{身高}{170cm}
\cvitem{婚姻状态}{未婚}
\cvitem{户口/国籍}{黑龙江省/中国}
\cvitem{现住址}{北京市海淀区甘家口街道三虎桥南路17号东院1号楼1单元}

\section{个人优势}
\cvitem{}{硬件终端开发经验,熟悉常用的芯片平台(Marvel/Atmel/Amlogic/MTK)及Android 系统的BSP/Multimedia
熟练使用C/C++语言,熟悉Java;  熟悉部分低速硬件的设计原理( 数字/模拟 ), 熟悉常用的硬件接口协议.
}

\section{期望职位}
\cvitem{Android 系统软件开发}{北京·\textbf{33K+}}
\cvitem{}{通信/网络设备·移动互联网·电子/半导体/集成电路}

\section{工作经验}
\textcolor{lgreen}{\rule[-7pt]{20.5cm}{0.1em}}
\subsection{Android (Multimedia)}
\cventry{2020/6--今}{WYZE中国-北京研发中心}{北京爱和健康科技有限公司}{}{}{
\textbf{工作描述:}
\begin{itemize}
   \item ffmpeg转码并保存DASH(H264 + G711)为MP4(H264 + AAC)
   \item ExoPlayer + ffmpeg实现软解G711音频支持, 熟悉ExoPlayer的编解码器扩展
   \item 完成ffmpeg高版本的交叉编译与集成(包括openh264/fdk-aac等), 配合GPUImage完成滤镜效果
   \item Google WebRTC SDK(Andorid)的集成与二次开发, 完成部分WebRTC在App端的界面设计, 完成WebRTC视频流的边播边存的业务需求
   \item 通过采取静态链接到libwebrtc.a的方式优化native库对WebRTC的设置和尺寸.
   \item 通过WebRTC的SurfaceViewRenderer实现RTC画面的缩放(OpenGL)
   \item GStreamer \& WebRTC插件RTC推流demo(Android App)
   \item 熟悉: ijkplayer/ExoPlayer/MediaPlayer/MeidaCodec/OMX IL/AudioFlinger/AudioPolicy等
   \item 熟悉NDK相关的开发流程, 熟悉Android Natvie代码的调试, 熟悉Framework的Java层调试
\end{itemize}
\textbf{项目经验:}
\begin{itemize}
\item WYZE App(Android)
   \begin{itemize}
      \item Camera相关插件, 与WebRTC技术有关的实现
      \item 部分NDK底层的数据加密/解密实现(基于libopenssl加密解密等)
      \item 部分NDK底层编解码库的集成(基于ffmpeg等)
      \item 新技术的调研:基于ExoPlayer的网络流播放器
      \item 基于GStramer(for Android)的下一代播放器设计
      \item 部分关于App的Jenkins打包Pipeline脚本(Groovy)
   \end{itemize}
\end{itemize}
}

\subsection{Android ROM开发}
\cventry{2019/1--2020/3}{声联网事业部}{北京小雅星空科技有限公司}{}{}{
\textbf{工作描述:}
\begin{itemize}
   \item BSP 驱动软件的集成, 产品研发过程中相关问题的解决与维护
   \item 积累了Audio Frameworks/tinyalsa/alsa driver等经验
   \item 熟悉Androiod相机应用Camera 2, Camera的API1/API2, Camera HAL1/HAL3等相关的部分, 熟悉NDK下的Camera 接口
   \item 调研学习了GStreamer多媒体框架, 调用Android Camear
   \item 积累了部分CI相关的经验, 如: Jenkins常见插件配置, Gerrit Code Review权限配置等
   \item 熟悉NDK相关的开发流程, 熟悉Android Natvie代码的调试(VSCode等工具)
\end{itemize}
\textbf{项目经验:}
\begin{itemize}
\item 触屏音箱
   \begin{itemize}
      \item BSP 驱动软件问题的解决与维护, 包括但不限于: Charge/LCM/Audio/PA/BT/WiFi/keypad
      \item 完成Jenkins代码编译的Pipeline, 优化编译速度, 完成编译后到OTA服务器的更新, 邮件通知等
      \item 驱动BUG修复, 问题分析/实验验证/集成与测试补丁
      \item 自动化测试,  树莓派 + Jenkins + 部分小硬件完成硬件的基本功能测试
      \item 系统应用的裁剪、集成, 包括但不限于对定位服务/
      \item OTA 服务的集成、验证, 与供应商对接系统升级流程, 修复recovery报错
   \end{itemize}
\end{itemize}
}

\subsection{Android BSP开发}
\cventry{2018/3--2018/12}{智能硬件部}{北京小米移动软件科技有限公司}{}{}{
\textbf{工作描述:}
\begin{itemize}
   \item 负责 Android 系统硬件的相关驱动:MIPI DSI LCD/Touch Panel/Keypads/PA等相关驱动
   \item 配合硬件开发完成产品的验证、测试
   \item 与第三方合作开发工厂测试 ROM
\end{itemize}
\textbf{项目经验: }
\begin{itemize}
   \item 小爱触屏音箱
   \begin{itemize}
      \item Camera/Ext PA/ALSPS/MIPI LCM/Touch/PMIC/Keypad 等驱动适配
      \item 积累了Camera Porting的相关经验,熟悉Camera的硬件电路链接,熟悉CCI/CSI的相关概念
      \item MTK SecureBoot v2.1/更新 dm-verity 证书/更新 platform 等证书/DA 相关工作
      \item OTA包全量/查量编译等, 积累了UpdateEngine/Bootctl的相关经验
      \item 工厂模式定制/集成工厂端测试 熟悉Factory init流程
   \end{itemize}
\end{itemize}
}

\subsection{Android 驱动工程师}
\cventry{2016/9--2017/12}{软件一部}{北京小鸟听听科技有限公司}{}{}{
\textbf{工作描述:}:
\begin{itemize}
   \item 负责 Audio ADC/DAC 驱动
   \item 负责 Android Recoery 二次开发(UI), 扩展 updater 功能, 实现系统升级 UI 的客制化
   \item 负责 MIPI/TP 驱动
\end{itemize}
\textbf{项目经验:}
\begin{itemize}
\item 触屏音箱
   \begin{itemize}
      \item 完成 Android 系统 OTA 定制, 扩展了 Recovery OTA 的脚本功能(通过扩展 edify 实现)
      \item 完成 ADC 驱动适配,DAC 驱动开发(TI)
      \item 完成 MIPI AMOLED 驱动的适配与调试
      \item 完成电容触摸屏的驱动适配
   \end{itemize}
\end{itemize}
}

\subsection{Linux 系统工程师}
\cventry{2014/6--2016/6}{软件部}{太仪(北京)科技有限公司}{}{}{
\textbf{工作描述:}
\begin{itemize}
   \item 负责 FPGA 驱动相关的开发工作
   \item 基于 Buildroot 构建 Bootloader/Linux Kernel/Ramdisk 整个操作系统
   \item 使用 libevent/libcurl/libjson-cpp/libconfig,构建应用端的运行框架
   \item 负责系统软件的整体构建, A/B 升级策略在 linux 系统下的实现
\end{itemize}
\textbf{项目经验:}
\begin{itemize}
   \item LTE 光纤拉远直放站项目
   \begin{itemize}
      \item 基于 Json-RPC-cpp 构建消息 Client -> Server 实现
      \item 完成 SIM300A GSM 模块的 3G 网络下的数据通信实现
      \item 解决业务程序在不同设备间的兼容性问题, 实现光模块功率, 状态的监控
      \item 负责全系统构建问题的解决
   \end{itemize}
\end{itemize}
}

\section{教育经历}
\cventry{2008/8--2012/07}{本科}{工业工程}{哈尔滨商业大学德强商务学院}{\textit{哈尔滨}}
{含电子工艺及管理方向}  % arguments 3 to 6 can be left empty

\section{兴趣爱好}
\cvitem{骑行}{公路骑行}
\cvitem{电子}{模拟电路等}

\end{document}